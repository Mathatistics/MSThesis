Many multinational companies and policy makers carry out decisions by speculating exchange rate. Exchange rate is determined by the demand and supply of a currency. It depends highly on variables like imports, exports, interest rates, oil prices, inflation and even with its past values. Since these macroeconomic variables are highly correlated with each other, latent variables or principal components can solve the problem of multicollinearity. The application of latent variables and principal components based methods such as Principal Component Regression (PCR) and Partial Least Square (PLS) in time series data for prediction is uncommon. Prediction of exchange rate of Norwegian Krone per Euro using Multiple linear regression, Principal Component Regression (PCR) and Partial Least Square (PLS) regression is performed in this dissertation.

Linear models and its subsets obtained using criteria such as minimum AIC or BIC and maximum $R^2$adj are compared on the basis of their goodness of fit. The selected model is then compared with models from principal component regression and partial least square regression on the basis of predictability criteria of RMSEP and $R^2$ predicted. The results have suggested the partial least square regression as the best models among other. The residuals obtained from the models have no autocorrelations so the application of this method has not only reduced the dimension of data but also resolved the problem of multicollinearity and autocorrelations.